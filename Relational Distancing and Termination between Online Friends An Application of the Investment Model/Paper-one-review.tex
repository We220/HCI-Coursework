\documentclass[12pt]{article}
\usepackage[english]{babel}
\usepackage[utf8]{inputenc}
\usepackage[autostyle]{csquotes}
\usepackage{natbib}
\usepackage{url}
\usepackage{hyperref}
\usepackage{todonotes}
\usepackage{titling}
\usepackage{subcaption}
\MakeOuterQuote{"}
\newcommand\cites[1]{\citeauthor{#1}'s\ (\citeyear{#1})}
\AtBeginDocument{
  \addtocontents{toc}{\small}
  \addtocontents{lof}{\small}
}
\title{A review of Relational Distancing and Termination between Online Friends: An Application of the Investment Model}
\author{William Ewart}
\date{April 2017}

\begin{document}
\begin{titlingpage}
\maketitle
\tableofcontents
\end{titlingpage}
\newpage
\section{Contribution}
Does the article add new information to the body of knowledge on the topic. What implication does the information have for how the topic is understood
Does it raise questions about accepted ways of understanding an issue, what is the most important question
How would readers think differently after reading the article
How might it be used in other work


\section{Justification of Conclusions}
In this section the way in which \cite{Carpenter} justifies the conclusions reached in the paper is critiqued. The first conclusion made is that a primary user is less likely to permanently cut ties with an annoyer. The more permanent the method to ignore the annoyer the less likely the primary user is to enact on it. This trend can be seen quite easily in the table that has collated all of the results from the questionnaire. They also use a ANOVA test to prove that the trend is linear and statistically significant. 
The next conclusion made was that the GE narcissism was related to the annoyers reports of their own behaviour. That is to say that annoyers generally exhibit GE narcissism which is the reason for posting self centred content. The paper claims that the responses given substantially support this hypothesis however the questionnaire given to the annoyers was measured such that a narcissistic reply was measured as one and a non-narcissistic reply was measured as zero. In the results the mean score for the annoyers was 0.4 which suggests that for the annoyers to substantially exhibit GE narcissism traits it would as least be over 0.5. Maybe there is more to the annoyers posting self-centred content than the paper claims. 
Another claim was that the primary users satisfaction with the friendship of the annoyer would be negatively related to the annoyers posting behaviour increasing. The results reported in the paper support this claim with a negative correlation of -0.34. There are no raw data values in the paper so the calculations that the writers have done are assumed to be correct.
The following claims by the paper are all in relation to the investment model and applying it to Facebook friends. The results reported support all of the claims made by \cite{Carpenter}. These are as follows; satisfaction with a friendship is positively related to the commitment of the friendship, investment in a friendship is positively associated with commitment to the friendship, the perceived quality of the primary users network of friends would be positively related to the assessments of other alternatives to the friends, the quality of alternatives would be negatively related to the commitment of the current friendships, commitment to friendships would be negatively related to intentions of unfriending, hiding and skipping the individuals content. 
However even though the data reported supported these findings when a model fit was used to test the paths of the hypothesised investment model for Facebook friends corrections had to be made. 

Were the adjustments made to the hypothesised investment model justified appropriately?

First model had a poor fit with the data
Modifications made to it the first being the intent to skip outcome. The relationship between it and commitment was lower than first thought. It was more related to the extent that the annoyer produced self focused messages. Not what the model predicted at first
Investment was also identified as problematic with the model. It does not play as large a role in computer communication and was removed.
After both of these variables were removed the model fits.
Talk about the method used to obtain results. The paper uses a questionnaire to gather information on users behaviour to annoyers. Maybe the behaviour could be observed rather than quizzed. 

\section{Further Work}
Research investigating the association of the investment model (IM) variables to the nuances of passive versus active behaviours is needed
Research how annoyers respond to the primary users using distancing tactics on their posts and judging them for what they post
Research more factors that encourage distancing over termination
Offline factors may play a part in distancing or terminating a friend
Intent measured rather than actual behaviour need to measure actual behaviour
Do a larger study on the annoyers. Only a small amount of them were surveyed. Behaviour was put down as narcissistic but not a lot of evidence to support this
\cite{Carpenter}

\bibliographystyle{agsm}
\bibliography{bibliography}

\end{document}